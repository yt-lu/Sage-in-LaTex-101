% !TEX TS-program = pdflatexmk

\documentclass{amsart}
\usepackage{sagetex}
\usepackage{xcolor, framed}

\begin{document}

Using Sage\TeX, one can use Sage to compute things and put them into
your \LaTeX{} document. For example, there are
$\sage{number_of_partitions(1269)}$ integer partitions of $1269$.
You don't need to compute the number elsewhere, then copy and paste
it into the paragraph. All you need is to call the command {\color{red}\verb|\sage{number_of_partitions(1269)}|}.
  
  
 To {\em execute and print} a block of sage commands, use
 {\color{red}
 \begin{sageverbatim}
 \begin{sageblock}
    f(x) = exp(x) * sin(2*x)
\end{sageblock}
 \end{sageverbatim}
}

To {\em execute and hide} a block of sage commands, use 
 {\color{red}
 \begin{sageverbatim}
 \begin{sagesilent}
    f(x) = exp(x) * sin(2*x)
\end{sagesilent}
 \end{sageverbatim}
}

To {\em only print} psuedo-codes, use
{\color{red}
 \begin{verbatim}
 \begin{sageverbatim}
    f(x) = exp(x) * sin(2*x)
\end{sageverbatim}
 \end{verbatim}
}

%=============Example: Derivative==========
\begin{framed}
For example, when you do 
 
 {\color{red}
 \begin{verbatim}
 \begin{sagesilent}
    f(x) = exp(x) * sin(2*x)
\end{sagesilent}

\[
  \frac{\mathrm{d}^{2}}{\mathrm{d}x^{2}} \sage{f(x)} =
  \sage{diff(f, x, 2)(x)}.
\]
\end{verbatim}
}
it produces this
 \begin{sagesilent}
    f(x) = exp(x) * sin(2*x)
\end{sagesilent}
\[
  \frac{\mathrm{d}^{2}}{\mathrm{d}x^{2}} \sage{f(x)} =
  \sage{diff(f, x, 2)(x)}.
\]

\end{framed}
To {\em plot} a graph, use {\color{red}\verb|\sageplot[options]{object, option, more options}|}, which can draw both 2D and 3D figures. 

%============Example:2d plot==========
\begin{framed}
Here's a 2D plot of $f$ (that we created earlier in this file) from $-1$ to $1$:

% Print pseudo-codes
{\color{red}
\begin{verbatim}
\begin{figure}[h]		
\sageplot[width=0.8\textwidth]{plot(f, -1, 1)}
\end{figure}
\end{verbatim}
}
\end{framed}

% Execute codes
\begin{figure}[h]
\sageplot[width=0.8\textwidth]{plot(f, -1, 1)}
\end{figure}


%============Example:3d plot==========
% Execute codes
and an example of a 3D graph. 
\begin{sagesilent}
var('s,t');
p=parametric_plot3d([s*cos(t), s*sin(t),s^2*(sin(t)^2 - cos(t)^2)], (s,0,1),(t, 0, 2*pi), opacity=0.8);
q= parametric_plot3d([0.5*cos(t), 0.5*sin(t), 0.25*(sin(t)^2 - cos(t)^2)], (t, 0, 2*pi), opacity=0.8, color='red', radius=0.01)
\end{sagesilent}

\begin{figure}[h]
\sageplot[width=0.8\textwidth]{p+q, frame=True}
\end{figure}
\end{document}